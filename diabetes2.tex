% Options for packages loaded elsewhere
\PassOptionsToPackage{unicode}{hyperref}
\PassOptionsToPackage{hyphens}{url}
\documentclass[
]{article}
\usepackage{xcolor}
\usepackage[margin=1in]{geometry}
\usepackage{amsmath,amssymb}
\setcounter{secnumdepth}{-\maxdimen} % remove section numbering
\usepackage{iftex}
\ifPDFTeX
  \usepackage[T1]{fontenc}
  \usepackage[utf8]{inputenc}
  \usepackage{textcomp} % provide euro and other symbols
\else % if luatex or xetex
  \usepackage{unicode-math} % this also loads fontspec
  \defaultfontfeatures{Scale=MatchLowercase}
  \defaultfontfeatures[\rmfamily]{Ligatures=TeX,Scale=1}
\fi
\usepackage{lmodern}
\ifPDFTeX\else
  % xetex/luatex font selection
\fi
% Use upquote if available, for straight quotes in verbatim environments
\IfFileExists{upquote.sty}{\usepackage{upquote}}{}
\IfFileExists{microtype.sty}{% use microtype if available
  \usepackage[]{microtype}
  \UseMicrotypeSet[protrusion]{basicmath} % disable protrusion for tt fonts
}{}
\makeatletter
\@ifundefined{KOMAClassName}{% if non-KOMA class
  \IfFileExists{parskip.sty}{%
    \usepackage{parskip}
  }{% else
    \setlength{\parindent}{0pt}
    \setlength{\parskip}{6pt plus 2pt minus 1pt}}
}{% if KOMA class
  \KOMAoptions{parskip=half}}
\makeatother
\usepackage{color}
\usepackage{fancyvrb}
\newcommand{\VerbBar}{|}
\newcommand{\VERB}{\Verb[commandchars=\\\{\}]}
\DefineVerbatimEnvironment{Highlighting}{Verbatim}{commandchars=\\\{\}}
% Add ',fontsize=\small' for more characters per line
\usepackage{framed}
\definecolor{shadecolor}{RGB}{248,248,248}
\newenvironment{Shaded}{\begin{snugshade}}{\end{snugshade}}
\newcommand{\AlertTok}[1]{\textcolor[rgb]{0.94,0.16,0.16}{#1}}
\newcommand{\AnnotationTok}[1]{\textcolor[rgb]{0.56,0.35,0.01}{\textbf{\textit{#1}}}}
\newcommand{\AttributeTok}[1]{\textcolor[rgb]{0.13,0.29,0.53}{#1}}
\newcommand{\BaseNTok}[1]{\textcolor[rgb]{0.00,0.00,0.81}{#1}}
\newcommand{\BuiltInTok}[1]{#1}
\newcommand{\CharTok}[1]{\textcolor[rgb]{0.31,0.60,0.02}{#1}}
\newcommand{\CommentTok}[1]{\textcolor[rgb]{0.56,0.35,0.01}{\textit{#1}}}
\newcommand{\CommentVarTok}[1]{\textcolor[rgb]{0.56,0.35,0.01}{\textbf{\textit{#1}}}}
\newcommand{\ConstantTok}[1]{\textcolor[rgb]{0.56,0.35,0.01}{#1}}
\newcommand{\ControlFlowTok}[1]{\textcolor[rgb]{0.13,0.29,0.53}{\textbf{#1}}}
\newcommand{\DataTypeTok}[1]{\textcolor[rgb]{0.13,0.29,0.53}{#1}}
\newcommand{\DecValTok}[1]{\textcolor[rgb]{0.00,0.00,0.81}{#1}}
\newcommand{\DocumentationTok}[1]{\textcolor[rgb]{0.56,0.35,0.01}{\textbf{\textit{#1}}}}
\newcommand{\ErrorTok}[1]{\textcolor[rgb]{0.64,0.00,0.00}{\textbf{#1}}}
\newcommand{\ExtensionTok}[1]{#1}
\newcommand{\FloatTok}[1]{\textcolor[rgb]{0.00,0.00,0.81}{#1}}
\newcommand{\FunctionTok}[1]{\textcolor[rgb]{0.13,0.29,0.53}{\textbf{#1}}}
\newcommand{\ImportTok}[1]{#1}
\newcommand{\InformationTok}[1]{\textcolor[rgb]{0.56,0.35,0.01}{\textbf{\textit{#1}}}}
\newcommand{\KeywordTok}[1]{\textcolor[rgb]{0.13,0.29,0.53}{\textbf{#1}}}
\newcommand{\NormalTok}[1]{#1}
\newcommand{\OperatorTok}[1]{\textcolor[rgb]{0.81,0.36,0.00}{\textbf{#1}}}
\newcommand{\OtherTok}[1]{\textcolor[rgb]{0.56,0.35,0.01}{#1}}
\newcommand{\PreprocessorTok}[1]{\textcolor[rgb]{0.56,0.35,0.01}{\textit{#1}}}
\newcommand{\RegionMarkerTok}[1]{#1}
\newcommand{\SpecialCharTok}[1]{\textcolor[rgb]{0.81,0.36,0.00}{\textbf{#1}}}
\newcommand{\SpecialStringTok}[1]{\textcolor[rgb]{0.31,0.60,0.02}{#1}}
\newcommand{\StringTok}[1]{\textcolor[rgb]{0.31,0.60,0.02}{#1}}
\newcommand{\VariableTok}[1]{\textcolor[rgb]{0.00,0.00,0.00}{#1}}
\newcommand{\VerbatimStringTok}[1]{\textcolor[rgb]{0.31,0.60,0.02}{#1}}
\newcommand{\WarningTok}[1]{\textcolor[rgb]{0.56,0.35,0.01}{\textbf{\textit{#1}}}}
\usepackage{longtable,booktabs,array}
\usepackage{calc} % for calculating minipage widths
% Correct order of tables after \paragraph or \subparagraph
\usepackage{etoolbox}
\makeatletter
\patchcmd\longtable{\par}{\if@noskipsec\mbox{}\fi\par}{}{}
\makeatother
% Allow footnotes in longtable head/foot
\IfFileExists{footnotehyper.sty}{\usepackage{footnotehyper}}{\usepackage{footnote}}
\makesavenoteenv{longtable}
\usepackage{graphicx}
\makeatletter
\newsavebox\pandoc@box
\newcommand*\pandocbounded[1]{% scales image to fit in text height/width
  \sbox\pandoc@box{#1}%
  \Gscale@div\@tempa{\textheight}{\dimexpr\ht\pandoc@box+\dp\pandoc@box\relax}%
  \Gscale@div\@tempb{\linewidth}{\wd\pandoc@box}%
  \ifdim\@tempb\p@<\@tempa\p@\let\@tempa\@tempb\fi% select the smaller of both
  \ifdim\@tempa\p@<\p@\scalebox{\@tempa}{\usebox\pandoc@box}%
  \else\usebox{\pandoc@box}%
  \fi%
}
% Set default figure placement to htbp
\def\fps@figure{htbp}
\makeatother
\setlength{\emergencystretch}{3em} % prevent overfull lines
\providecommand{\tightlist}{%
  \setlength{\itemsep}{0pt}\setlength{\parskip}{0pt}}
\usepackage{bookmark}
\IfFileExists{xurl.sty}{\usepackage{xurl}}{} % add URL line breaks if available
\urlstyle{same}
\hypersetup{
  pdftitle={Develop a Computational Phenotyping Algorithm to Identify Patients with Type II diabetes},
  hidelinks,
  pdfcreator={LaTeX via pandoc}}

\title{Develop a Computational Phenotyping Algorithm to Identify
Patients with Type II diabetes}
\author{}
\date{\vspace{-2.5em}}

\begin{document}
\maketitle

{
\setcounter{tocdepth}{2}
\tableofcontents
}
\subsubsection{Background on Diabetes}\label{background-on-diabetes}

Type II diabetes is a type of diabetes that is caused by the body no
longer recognizing and appropriately responding to insulin.

\paragraph{Diagnostic Criteria}\label{diagnostic-criteria}

\begin{itemize}
\tightlist
\item
  A fasting plasma glucose level of 126 mg/dL (7.0 mmol/L) or higher
  \textbf{OR}
\item
  A 2-hour plasma glucose level of 200 mg/dL (11.1 mmol/L) or higher
  during a 75g OGTT \textbf{OR}
\item
  A random plasma glucose of 200 mg/dL (11.1 mmol/L) or higher +
  symptoms \textbf{OR}
\item
  HbA1c of 6.5\% or higher
\end{itemize}

\paragraph{Treatments}\label{treatments}

\begin{itemize}
\tightlist
\item
  Metformin
\item
  Sulfonylureas (e.g., glyburide, glimepiride\ldots)
\item
  Thiazolidinediones (e.g., pioglitazone\ldots)
\item
  DPP-4 inhibitors (e.g., sitagliptin\ldots)
\item
  SGLT2 inhibitors (e.g., canagliflozin\ldots)
\item
  GLP-1 receptor agonists (e.g., liraglutide\ldots)
\end{itemize}

\paragraph{Laboratory Tests}\label{laboratory-tests}

\begin{itemize}
\tightlist
\item
  Plasma glucose (fasting / random / 2-hour OGTT)
\item
  HbA1c
\end{itemize}

\subsubsection{Set up connection to the Google BigQuery
project}\label{set-up-connection-to-the-google-bigquery-project}

\subsubsection{Load diababetes
goldstandard}\label{load-diababetes-goldstandard}

\begin{Shaded}
\begin{Highlighting}[]
\NormalTok{diabetes }\OtherTok{\textless{}{-}} \FunctionTok{bq\_project\_query}\NormalTok{(}
\NormalTok{  my\_project,}
  \StringTok{"}
\StringTok{  SELECT SUBJECT\_ID, DIABETES}
\StringTok{  FROM \textasciigrave{}course3\_data.diabetes\_goldstandard\textasciigrave{}}
\StringTok{  "}
\NormalTok{) }\SpecialCharTok{\%\textgreater{}\%} \FunctionTok{bq\_table\_download}\NormalTok{()}

\NormalTok{knitr}\SpecialCharTok{::}\FunctionTok{kable}\NormalTok{(diabetes)}
\end{Highlighting}
\end{Shaded}

\begin{longtable}[]{@{}rr@{}}
\toprule\noalign{}
SUBJECT\_ID & DIABETES \\
\midrule\noalign{}
\endhead
\bottomrule\noalign{}
\endlastfoot
10011 & 0 \\
10013 & 0 \\
10026 & 0 \\
10036 & 0 \\
10038 & 0 \\
10040 & 0 \\
10044 & 0 \\
10045 & 0 \\
10046 & 0 \\
10056 & 0 \\
10065 & 0 \\
10083 & 0 \\
10098 & 0 \\
10112 & 0 \\
10117 & 0 \\
10126 & 0 \\
10127 & 0 \\
40124 & 0 \\
40277 & 0 \\
40286 & 0 \\
40601 & 0 \\
40456 & 0 \\
40595 & 0 \\
40612 & 0 \\
40687 & 0 \\
41983 & 0 \\
42231 & 0 \\
42281 & 0 \\
42321 & 0 \\
43746 & 0 \\
43827 & 0 \\
43879 & 0 \\
43909 & 0 \\
44228 & 0 \\
44212 & 0 \\
10019 & 0 \\
10029 & 0 \\
10032 & 0 \\
10035 & 0 \\
10042 & 0 \\
10043 & 0 \\
10059 & 0 \\
10064 & 0 \\
10067 & 0 \\
10074 & 0 \\
10076 & 0 \\
10088 & 0 \\
10089 & 0 \\
10090 & 0 \\
10093 & 0 \\
10101 & 0 \\
10102 & 0 \\
10119 & 0 \\
10120 & 0 \\
40310 & 0 \\
40304 & 0 \\
42075 & 0 \\
42066 & 0 \\
42135 & 0 \\
42275 & 0 \\
43748 & 0 \\
42412 & 0 \\
42458 & 0 \\
43881 & 0 \\
44154 & 0 \\
10006 & 1 \\
10017 & 1 \\
10027 & 1 \\
10033 & 1 \\
10061 & 1 \\
10069 & 1 \\
10104 & 1 \\
10111 & 1 \\
10114 & 1 \\
10124 & 1 \\
10132 & 1 \\
40503 & 1 \\
40655 & 1 \\
42033 & 1 \\
42199 & 1 \\
42302 & 1 \\
42346 & 1 \\
42367 & 1 \\
43870 & 1 \\
43927 & 1 \\
10094 & 1 \\
10106 & 1 \\
10130 & 1 \\
40177 & 1 \\
40204 & 1 \\
41795 & 1 \\
41914 & 1 \\
41976 & 1 \\
42292 & 1 \\
42430 & 1 \\
43779 & 1 \\
43798 & 1 \\
44083 & 1 \\
44222 & 1 \\
\end{longtable}

In this table the DIABETES column is a 1 if the patient has a record of
type II diabetes and a 0 if they did not have the condition.Of the 100
patients in the demo data set, 99 had notes that could be reviewed. Of
those 99 records reviewed, 34 had type II diabetes.

\paragraph{Querying and Assessing ICD
codes}\label{querying-and-assessing-icd-codes}

\textbf{There are many ICD-9 codes for diabetes:}

\begin{longtable}[]{@{}
  >{\raggedright\arraybackslash}p{(\linewidth - 2\tabcolsep) * \real{0.6250}}
  >{\raggedright\arraybackslash}p{(\linewidth - 2\tabcolsep) * \real{0.3750}}@{}}
\toprule\noalign{}
\begin{minipage}[b]{\linewidth}\raggedright
ICD-9 Code
\end{minipage} & \begin{minipage}[b]{\linewidth}\raggedright
Label
\end{minipage} \\
\midrule\noalign{}
\endhead
\bottomrule\noalign{}
\endlastfoot
250 & Diabetes mellitus \\
250.0 & Diabetes mellitus without mention of complication \\
250.00 & Diabetes mellitus without mention of complication, type II or
unspecified type, not stated as uncontrolled \\
250.01 & Diabetes mellitus without mention of complication, type I
(juvenile type), not stated as uncontrolled \\
250.02 & Diabetes mellitus without mention of complication, type II or
unspecified type, uncontrolled \\
\end{longtable}

\begin{Shaded}
\begin{Highlighting}[]
\CommentTok{\#Load the tables }
\NormalTok{training }\OtherTok{\textless{}{-}} \FunctionTok{tbl}\NormalTok{(con, }\StringTok{"course3\_data.diabetes\_training"}\NormalTok{)}
\NormalTok{diagnoses\_icd }\OtherTok{\textless{}{-}} \FunctionTok{tbl}\NormalTok{(con, }\StringTok{"mimic3\_demo.DIAGNOSES\_ICD"}\NormalTok{)}

\CommentTok{\#Identify the patients with ICD\_25000}
\NormalTok{icd\_25000 }\OtherTok{\textless{}{-}}\NormalTok{ diagnoses\_icd }\SpecialCharTok{\%\textgreater{}\%} 
  \FunctionTok{filter}\NormalTok{(ICD9\_CODE }\SpecialCharTok{==} \StringTok{"25000"}\NormalTok{) }\SpecialCharTok{\%\textgreater{}\%} 
  \FunctionTok{distinct}\NormalTok{(SUBJECT\_ID) }\SpecialCharTok{\%\textgreater{}\%} 
  \FunctionTok{mutate}\NormalTok{(}\AttributeTok{icd\_25000 =} \DecValTok{1}\NormalTok{)}
\NormalTok{knitr}\SpecialCharTok{::}\FunctionTok{kable}\NormalTok{(}\FunctionTok{head}\NormalTok{(icd\_25000,}\DecValTok{10}\NormalTok{))}
\end{Highlighting}
\end{Shaded}

\begin{longtable}[]{@{}rr@{}}
\toprule\noalign{}
SUBJECT\_ID & icd\_25000 \\
\midrule\noalign{}
\endhead
\bottomrule\noalign{}
\endlastfoot
10106 & 1 \\
43779 & 1 \\
40204 & 1 \\
10006 & 1 \\
43798 & 1 \\
10017 & 1 \\
10027 & 1 \\
10033 & 1 \\
40503 & 1 \\
10045 & 1 \\
\end{longtable}

\begin{Shaded}
\begin{Highlighting}[]
\CommentTok{\# Join icd\_25000 with the diabetes data frame }
\NormalTok{training\_joined }\OtherTok{\textless{}{-}}\NormalTok{ training }\SpecialCharTok{\%\textgreater{}\%}
  \FunctionTok{left\_join}\NormalTok{(icd\_25000, }\AttributeTok{by =} \StringTok{"SUBJECT\_ID"}\NormalTok{) }\SpecialCharTok{\%\textgreater{}\%}
  \FunctionTok{mutate}\NormalTok{(}\AttributeTok{icd\_25000 =} \FunctionTok{coalesce}\NormalTok{(icd\_25000, }\DecValTok{0}\NormalTok{))}

\NormalTok{knitr}\SpecialCharTok{::}\FunctionTok{kable}\NormalTok{(}\FunctionTok{head}\NormalTok{(training\_joined,}\DecValTok{20}\NormalTok{))}
\end{Highlighting}
\end{Shaded}

\begin{longtable}[]{@{}rrr@{}}
\toprule\noalign{}
SUBJECT\_ID & DIABETES & icd\_25000 \\
\midrule\noalign{}
\endhead
\bottomrule\noalign{}
\endlastfoot
10026 & 0 & 0 \\
40310 & 0 & 0 \\
10067 & 0 & 0 \\
44228 & 0 & 0 \\
10064 & 0 & 0 \\
10126 & 0 & 0 \\
10102 & 0 & 0 \\
10045 & 0 & 1 \\
42231 & 0 & 0 \\
10065 & 0 & 0 \\
40612 & 0 & 0 \\
10043 & 0 & 0 \\
40124 & 0 & 0 \\
10032 & 0 & 0 \\
42321 & 0 & 0 \\
10076 & 0 & 0 \\
10093 & 0 & 0 \\
42275 & 0 & 0 \\
40601 & 0 & 0 \\
10040 & 0 & 0 \\
\end{longtable}

\begin{Shaded}
\begin{Highlighting}[]
\CommentTok{\# Function to Calculate Performance}
\FunctionTok{library}\NormalTok{(caret)}
\NormalTok{getStats }\OtherTok{\textless{}{-}} \ControlFlowTok{function}\NormalTok{(df, ...) \{}
\NormalTok{  df }\SpecialCharTok{\%\textgreater{}\%}
    \FunctionTok{select}\NormalTok{(...) }\SpecialCharTok{\%\textgreater{}\%}
    \FunctionTok{mutate}\NormalTok{(}\FunctionTok{across}\NormalTok{(}\FunctionTok{everything}\NormalTok{(), }\SpecialCharTok{\textasciitilde{}} \FunctionTok{factor}\NormalTok{(.x, }\AttributeTok{levels =} \FunctionTok{c}\NormalTok{(}\DecValTok{1}\NormalTok{, }\DecValTok{0}\NormalTok{)))) }\SpecialCharTok{\%\textgreater{}\%}
    \FunctionTok{table}\NormalTok{() }\SpecialCharTok{\%\textgreater{}\%}
    \FunctionTok{confusionMatrix}\NormalTok{()}
\NormalTok{\}}
\end{Highlighting}
\end{Shaded}

\begin{Shaded}
\begin{Highlighting}[]
\CommentTok{\# Calculate the performance of icd\_25000}
\NormalTok{training }\SpecialCharTok{\%\textless{}\textgreater{}\%} 
  \FunctionTok{left\_join}\NormalTok{(icd\_25000) }\SpecialCharTok{\%\textgreater{}\%} 
  \FunctionTok{mutate}\NormalTok{(}\AttributeTok{icd\_25000 =} \FunctionTok{coalesce}\NormalTok{(icd\_25000, }\DecValTok{0}\NormalTok{))}
\NormalTok{training }\SpecialCharTok{\%\textgreater{}\%} 
  \FunctionTok{collect}\NormalTok{() }\SpecialCharTok{\%\textgreater{}\%} 
  \FunctionTok{getStats}\NormalTok{(icd\_25000, DIABETES)}
\end{Highlighting}
\end{Shaded}

\begin{verbatim}
## Confusion Matrix and Statistics
## 
##          DIABETES
## icd_25000  1  0
##         1 19  1
##         0  8 52
##                                           
##                Accuracy : 0.8875          
##                  95% CI : (0.7972, 0.9472)
##     No Information Rate : 0.6625          
##     P-Value [Acc > NIR] : 3.476e-06       
##                                           
##                   Kappa : 0.7313          
##                                           
##  Mcnemar's Test P-Value : 0.0455          
##                                           
##             Sensitivity : 0.7037          
##             Specificity : 0.9811          
##          Pos Pred Value : 0.9500          
##          Neg Pred Value : 0.8667          
##              Prevalence : 0.3375          
##          Detection Rate : 0.2375          
##    Detection Prevalence : 0.2500          
##       Balanced Accuracy : 0.8424          
##                                           
##        'Positive' Class : 1               
## 
\end{verbatim}

\subparagraph{This code actually performs fairly well. ICD9 250.00 has a
decent specificity of 98.11\%. However the sensitivity is not great at
only
70.37\%.}\label{this-code-actually-performs-fairly-well.-icd9-250.00-has-a-decent-specificity-of-98.11.-however-the-sensitivity-is-not-great-at-only-70.37.}

\subsubsection{Querying and Assessing laboratory
data}\label{querying-and-assessing-laboratory-data}

\subparagraph{\texorpdfstring{As described in the introduction, there
are a number of laboratory tests used to diagnose diabetes.We will take
a look at just Hemoglobin A1C. MIMIC-III records lab tests with a
variety of labels. We can search these labels in the D\_LABITEMS table.
The following SQL query was executed in
\textbf{BigQuery}:}{As described in the introduction, there are a number of laboratory tests used to diagnose diabetes.We will take a look at just Hemoglobin A1C. MIMIC-III records lab tests with a variety of labels. We can search these labels in the D\_LABITEMS table. The following SQL query was executed in BigQuery:}}\label{as-described-in-the-introduction-there-are-a-number-of-laboratory-tests-used-to-diagnose-diabetes.we-will-take-a-look-at-just-hemoglobin-a1c.-mimic-iii-records-lab-tests-with-a-variety-of-labels.-we-can-search-these-labels-in-the-d_labitems-table.-the-following-sql-query-was-executed-in-bigquery}

select * from mimic3\_demo.D\_LABITEMS where lower(LABEL) like
``\%a1c\%''

The results were:

\begin{longtable}[]{@{}ll@{}}
\toprule\noalign{}
ITEMID & LABEL \\
\midrule\noalign{}
\endhead
\bottomrule\noalign{}
\endlastfoot
50852 & \% Hemoglobin A1c \\
50854 & Absolute A1c \\
\end{longtable}

\begin{Shaded}
\begin{Highlighting}[]
\CommentTok{\# Load labevents table}
\NormalTok{labevents }\OtherTok{\textless{}{-}} \FunctionTok{tbl}\NormalTok{(con, }\StringTok{"mimic3\_demo.LABEVENTS"}\NormalTok{)}

\CommentTok{\# Identify patients with hba1c}
\NormalTok{hba1c }\OtherTok{\textless{}{-}}\NormalTok{ labevents }\SpecialCharTok{\%\textgreater{}\%} 
  \FunctionTok{filter}\NormalTok{(ITEMID }\SpecialCharTok{\%in\%} \FunctionTok{c}\NormalTok{(}\DecValTok{50852}\NormalTok{,}\DecValTok{50854}\NormalTok{)) }\SpecialCharTok{\%\textgreater{}\%} 
  \FunctionTok{distinct}\NormalTok{(SUBJECT\_ID) }\SpecialCharTok{\%\textgreater{}\%} 
  \FunctionTok{mutate}\NormalTok{(}\AttributeTok{hba1c =} \DecValTok{1}\NormalTok{)}

\CommentTok{\# Merge hba1c indicator into training dataset}
\NormalTok{training }\SpecialCharTok{\%\textless{}\textgreater{}\%} 
  \FunctionTok{left\_join}\NormalTok{(hba1c) }\SpecialCharTok{\%\textgreater{}\%} 
  \FunctionTok{mutate}\NormalTok{(}\AttributeTok{hba1c =} \FunctionTok{coalesce}\NormalTok{(hba1c, }\DecValTok{0}\NormalTok{))}

\CommentTok{\# Evaluate performance}
\NormalTok{training }\SpecialCharTok{\%\textgreater{}\%} 
  \FunctionTok{collect}\NormalTok{() }\SpecialCharTok{\%\textgreater{}\%} 
  \FunctionTok{getStats}\NormalTok{(hba1c, DIABETES)}
\end{Highlighting}
\end{Shaded}

\begin{verbatim}
## Confusion Matrix and Statistics
## 
##      DIABETES
## hba1c  1  0
##     1  7  3
##     0 20 50
##                                           
##                Accuracy : 0.7125          
##                  95% CI : (0.6005, 0.8082)
##     No Information Rate : 0.6625          
##     P-Value [Acc > NIR] : 0.2051844       
##                                           
##                   Kappa : 0.2397          
##                                           
##  Mcnemar's Test P-Value : 0.0008492       
##                                           
##             Sensitivity : 0.2593          
##             Specificity : 0.9434          
##          Pos Pred Value : 0.7000          
##          Neg Pred Value : 0.7143          
##              Prevalence : 0.3375          
##          Detection Rate : 0.0875          
##    Detection Prevalence : 0.1250          
##       Balanced Accuracy : 0.6013          
##                                           
##        'Positive' Class : 1               
## 
\end{verbatim}

\subparagraph{The combined HbA1c labs have a moderate specificity of
94.34\%. However the sensitivity is very poor at only
25.93\%.}\label{the-combined-hba1c-labs-have-a-moderate-specificity-of-94.34.-however-the-sensitivity-is-very-poor-at-only-25.93.}

\subsubsection{Querying and Assessing Medication
data}\label{querying-and-assessing-medication-data}

As described in the introduction, there are a number of medications used
to treat diabetes. Let's try the first-line treatment metformin.

\begin{Shaded}
\begin{Highlighting}[]
\CommentTok{\# Load PRESCRIPTIONS table}
\NormalTok{prescriptions }\OtherTok{\textless{}{-}} \FunctionTok{tbl}\NormalTok{(con, }\StringTok{"mimic3\_demo.PRESCRIPTIONS"}\NormalTok{)}

\CommentTok{\# Identify patients with Metformin prescriptions}
\NormalTok{metformin }\OtherTok{\textless{}{-}}\NormalTok{ prescriptions }\SpecialCharTok{\%\textgreater{}\%} 
  \FunctionTok{filter}\NormalTok{(}\FunctionTok{tolower}\NormalTok{(DRUG) }\SpecialCharTok{\%like\%} \StringTok{"\%metformin\%"}\NormalTok{) }\SpecialCharTok{\%\textgreater{}\%} 
  \FunctionTok{distinct}\NormalTok{(SUBJECT\_ID) }\SpecialCharTok{\%\textgreater{}\%} 
  \FunctionTok{mutate}\NormalTok{(}\AttributeTok{metformin =} \DecValTok{1}\NormalTok{)}
\NormalTok{knitr}\SpecialCharTok{::}\FunctionTok{kable}\NormalTok{(}\FunctionTok{head}\NormalTok{(metformin,}\DecValTok{20}\NormalTok{))}
\end{Highlighting}
\end{Shaded}

\begin{longtable}[]{@{}rr@{}}
\toprule\noalign{}
SUBJECT\_ID & metformin \\
\midrule\noalign{}
\endhead
\bottomrule\noalign{}
\endlastfoot
10104 & 1 \\
10106 & 1 \\
43927 & 1 \\
\end{longtable}

\begin{Shaded}
\begin{Highlighting}[]
\CommentTok{\# Merge Metformin indicator into training dataset}
\NormalTok{training\_metformin }\OtherTok{\textless{}{-}}\NormalTok{ training }\SpecialCharTok{\%\textgreater{}\%} 
    \FunctionTok{left\_join}\NormalTok{(metformin, }\AttributeTok{by =} \StringTok{"SUBJECT\_ID"}\NormalTok{) }\SpecialCharTok{\%\textgreater{}\%} 
    \FunctionTok{mutate}\NormalTok{(}\AttributeTok{metformin =} \FunctionTok{coalesce}\NormalTok{(metformin, }\DecValTok{0}\NormalTok{))}
\NormalTok{knitr}\SpecialCharTok{::}\FunctionTok{kable}\NormalTok{(}\FunctionTok{head}\NormalTok{(training\_metformin,}\DecValTok{20}\NormalTok{))}
\end{Highlighting}
\end{Shaded}

\begin{longtable}[]{@{}rrrrr@{}}
\toprule\noalign{}
SUBJECT\_ID & DIABETES & icd\_25000 & hba1c & metformin \\
\midrule\noalign{}
\endhead
\bottomrule\noalign{}
\endlastfoot
10026 & 0 & 0 & 0 & 0 \\
40310 & 0 & 0 & 1 & 0 \\
10067 & 0 & 0 & 0 & 0 \\
44228 & 0 & 0 & 0 & 0 \\
10064 & 0 & 0 & 0 & 0 \\
10126 & 0 & 0 & 0 & 0 \\
10102 & 0 & 0 & 0 & 0 \\
10045 & 0 & 1 & 0 & 0 \\
42231 & 0 & 0 & 0 & 0 \\
10065 & 0 & 0 & 0 & 0 \\
40612 & 0 & 0 & 0 & 0 \\
10043 & 0 & 0 & 0 & 0 \\
40124 & 0 & 0 & 0 & 0 \\
10032 & 0 & 0 & 0 & 0 \\
42321 & 0 & 0 & 0 & 0 \\
10076 & 0 & 0 & 0 & 0 \\
10093 & 0 & 0 & 0 & 0 \\
42275 & 0 & 0 & 0 & 0 \\
40601 & 0 & 0 & 1 & 0 \\
10040 & 0 & 0 & 0 & 0 \\
\end{longtable}

\begin{Shaded}
\begin{Highlighting}[]
\CommentTok{\# Evaluate performance}
\NormalTok{training\_metformin }\SpecialCharTok{\%\textgreater{}\%} 
    \FunctionTok{collect}\NormalTok{() }\SpecialCharTok{\%\textgreater{}\%} 
    \FunctionTok{getStats}\NormalTok{(metformin, DIABETES)}
\end{Highlighting}
\end{Shaded}

\begin{verbatim}
## Confusion Matrix and Statistics
## 
##          DIABETES
## metformin  1  0
##         1  2  0
##         0 25 53
##                                           
##                Accuracy : 0.6875          
##                  95% CI : (0.5741, 0.7865)
##     No Information Rate : 0.6625          
##     P-Value [Acc > NIR] : 0.3658          
##                                           
##                   Kappa : 0.0958          
##                                           
##  Mcnemar's Test P-Value : 1.587e-06       
##                                           
##             Sensitivity : 0.07407         
##             Specificity : 1.00000         
##          Pos Pred Value : 1.00000         
##          Neg Pred Value : 0.67949         
##              Prevalence : 0.33750         
##          Detection Rate : 0.02500         
##    Detection Prevalence : 0.02500         
##       Balanced Accuracy : 0.53704         
##                                           
##        'Positive' Class : 1               
## 
\end{verbatim}

\subparagraph{Metformin has a perfect specificity of 100\%. However the
sensitivity is exceptionally poor at only 7.41\%. This is likely due to
the fact that most hospitalized patients are transitioned to insulin
during their hospital
stay.}\label{metformin-has-a-perfect-specificity-of-100.-however-the-sensitivity-is-exceptionally-poor-at-only-7.41.-this-is-likely-due-to-the-fact-that-most-hospitalized-patients-are-transitioned-to-insulin-during-their-hospital-stay.}

\subsubsection{Querying and Assessing the mean value of blood glucose
blood gas of at least 200
mg/dL}\label{querying-and-assessing-the-mean-value-of-blood-glucose-blood-gas-of-at-least-200-mgdl}

\begin{Shaded}
\begin{Highlighting}[]
\CommentTok{\# 1) Create a binary feature: mean blood gas glucose \textgreater{}= 200}
\NormalTok{labevents }\OtherTok{\textless{}{-}} \FunctionTok{tbl}\NormalTok{(con, }\StringTok{"mimic3\_demo.LABEVENTS"}\NormalTok{)}
\NormalTok{d\_labitems }\OtherTok{\textless{}{-}} \FunctionTok{tbl}\NormalTok{(con, }\StringTok{"mimic3\_demo.D\_LABITEMS"}\NormalTok{)}
\NormalTok{mean\_glucose\_blood\_bg\_over200 }\OtherTok{\textless{}{-}}\NormalTok{ labevents }\SpecialCharTok{\%\textgreater{}\%} 
  \FunctionTok{inner\_join}\NormalTok{(d\_labitems, }\AttributeTok{by =} \StringTok{"ITEMID"}\NormalTok{, }\AttributeTok{suffix =} \FunctionTok{c}\NormalTok{(}\StringTok{"\_l"}\NormalTok{,}\StringTok{"\_d"}\NormalTok{)) }\SpecialCharTok{\%\textgreater{}\%} 
  \FunctionTok{filter}\NormalTok{(}
\NormalTok{    LABEL }\SpecialCharTok{==} \StringTok{"Glucose"}\NormalTok{,}
\NormalTok{    FLUID }\SpecialCharTok{==} \StringTok{"Blood"}\NormalTok{,}
\NormalTok{    CATEGORY }\SpecialCharTok{==} \StringTok{"Blood Gas"}
\NormalTok{  ) }\SpecialCharTok{\%\textgreater{}\%} 
  \FunctionTok{group\_by}\NormalTok{(SUBJECT\_ID) }\SpecialCharTok{\%\textgreater{}\%} 
  \FunctionTok{summarise}\NormalTok{(}\AttributeTok{glucose\_blood\_bg\_mean =} \FunctionTok{mean}\NormalTok{(VALUENUM, }\AttributeTok{na.rm =} \ConstantTok{TRUE}\NormalTok{), }\AttributeTok{.groups =} \StringTok{"drop"}\NormalTok{) }\SpecialCharTok{\%\textgreater{}\%} 
  \FunctionTok{mutate}\NormalTok{(}\AttributeTok{mean\_glucose\_blood\_bg\_over200 =} \FunctionTok{if\_else}\NormalTok{(glucose\_blood\_bg\_mean }\SpecialCharTok{\textgreater{}=} \DecValTok{200}\NormalTok{, }\DecValTok{1}\DataTypeTok{L}\NormalTok{, }\DecValTok{0}\DataTypeTok{L}\NormalTok{)) }\SpecialCharTok{\%\textgreater{}\%} 
  \FunctionTok{select}\NormalTok{(SUBJECT\_ID, glucose\_blood\_bg\_mean, mean\_glucose\_blood\_bg\_over200)}
\NormalTok{knitr}\SpecialCharTok{::}\FunctionTok{kable}\NormalTok{(}\FunctionTok{head}\NormalTok{(mean\_glucose\_blood\_bg\_over200,}\DecValTok{20}\NormalTok{))}
\end{Highlighting}
\end{Shaded}

\begin{longtable}[]{@{}rrr@{}}
\toprule\noalign{}
SUBJECT\_ID & glucose\_blood\_bg\_mean &
mean\_glucose\_blood\_bg\_over200 \\
\midrule\noalign{}
\endhead
\bottomrule\noalign{}
\endlastfoot
40204 & 91.5000 & 0 \\
10126 & 122.9600 & 0 \\
10027 & 126.6970 & 0 \\
10093 & 61.0000 & 0 \\
41976 & 206.0000 & 1 \\
10059 & 129.5000 & 0 \\
10006 & 77.0000 & 0 \\
10019 & 163.3333 & 0 \\
42135 & 106.6000 & 0 \\
40595 & 161.3333 & 0 \\
10045 & 136.0625 & 0 \\
10042 & 143.1500 & 0 \\
42075 & 113.0000 & 0 \\
10120 & 289.6000 & 1 \\
10111 & 113.8000 & 0 \\
42292 & 92.0000 & 0 \\
10061 & 144.5000 & 0 \\
10065 & 135.2857 & 0 \\
10127 & 116.0000 & 0 \\
43927 & 130.1429 & 0 \\
\end{longtable}

\begin{Shaded}
\begin{Highlighting}[]
\CommentTok{\# 2) Join into training + fill missing with 0}
\NormalTok{training\_glucose }\OtherTok{\textless{}{-}}\NormalTok{ training }\SpecialCharTok{\%\textgreater{}\%} 
  \FunctionTok{left\_join}\NormalTok{(mean\_glucose\_blood\_bg\_over200, }\AttributeTok{by =} \StringTok{"SUBJECT\_ID"}\NormalTok{) }\SpecialCharTok{\%\textgreater{}\%} 
  \FunctionTok{mutate}\NormalTok{(}\AttributeTok{mean\_glucose\_blood\_bg\_over200 =} \FunctionTok{coalesce}\NormalTok{(mean\_glucose\_blood\_bg\_over200, }\DecValTok{0}\DataTypeTok{L}\NormalTok{))}
\NormalTok{knitr}\SpecialCharTok{::}\FunctionTok{kable}\NormalTok{(}\FunctionTok{head}\NormalTok{(training\_glucose,}\DecValTok{20}\NormalTok{))}
\end{Highlighting}
\end{Shaded}

\begin{longtable}[]{@{}
  >{\raggedleft\arraybackslash}p{(\linewidth - 10\tabcolsep) * \real{0.1250}}
  >{\raggedleft\arraybackslash}p{(\linewidth - 10\tabcolsep) * \real{0.1023}}
  >{\raggedleft\arraybackslash}p{(\linewidth - 10\tabcolsep) * \real{0.1136}}
  >{\raggedleft\arraybackslash}p{(\linewidth - 10\tabcolsep) * \real{0.0682}}
  >{\raggedleft\arraybackslash}p{(\linewidth - 10\tabcolsep) * \real{0.2500}}
  >{\raggedleft\arraybackslash}p{(\linewidth - 10\tabcolsep) * \real{0.3409}}@{}}
\toprule\noalign{}
\begin{minipage}[b]{\linewidth}\raggedleft
SUBJECT\_ID
\end{minipage} & \begin{minipage}[b]{\linewidth}\raggedleft
DIABETES
\end{minipage} & \begin{minipage}[b]{\linewidth}\raggedleft
icd\_25000
\end{minipage} & \begin{minipage}[b]{\linewidth}\raggedleft
hba1c
\end{minipage} & \begin{minipage}[b]{\linewidth}\raggedleft
glucose\_blood\_bg\_mean
\end{minipage} & \begin{minipage}[b]{\linewidth}\raggedleft
mean\_glucose\_blood\_bg\_over200
\end{minipage} \\
\midrule\noalign{}
\endhead
\bottomrule\noalign{}
\endlastfoot
10026 & 0 & 0 & 0 & NA & 0 \\
40310 & 0 & 0 & 1 & 107.6667 & 0 \\
10067 & 0 & 0 & 0 & 108.0000 & 0 \\
44228 & 0 & 0 & 0 & NA & 0 \\
10064 & 0 & 0 & 0 & 282.5000 & 1 \\
10126 & 0 & 0 & 0 & 122.9600 & 0 \\
10102 & 0 & 0 & 0 & NA & 0 \\
10045 & 0 & 1 & 0 & 136.0625 & 0 \\
42231 & 0 & 0 & 0 & 147.5000 & 0 \\
10065 & 0 & 0 & 0 & 135.2857 & 0 \\
40612 & 0 & 0 & 0 & 118.0000 & 0 \\
10043 & 0 & 0 & 0 & NA & 0 \\
40124 & 0 & 0 & 0 & NA & 0 \\
10032 & 0 & 0 & 0 & NA & 0 \\
42321 & 0 & 0 & 0 & NA & 0 \\
10076 & 0 & 0 & 0 & 136.5000 & 0 \\
10093 & 0 & 0 & 0 & 61.0000 & 0 \\
42275 & 0 & 0 & 0 & 150.0000 & 0 \\
40601 & 0 & 0 & 1 & NA & 0 \\
10040 & 0 & 0 & 0 & 110.0000 & 0 \\
\end{longtable}

\begin{Shaded}
\begin{Highlighting}[]
\CommentTok{\# 3) Collect to local and run getStats}
\NormalTok{training\_glucose\_local }\OtherTok{\textless{}{-}}\NormalTok{ training\_glucose }\SpecialCharTok{\%\textgreater{}\%} \FunctionTok{collect}\NormalTok{()}

\NormalTok{cm }\OtherTok{\textless{}{-}}\NormalTok{ training\_glucose\_local }\SpecialCharTok{\%\textgreater{}\%} 
  \FunctionTok{getStats}\NormalTok{(mean\_glucose\_blood\_bg\_over200, DIABETES)}

\NormalTok{cm   }\CommentTok{\# prints confusion matrix + stats}
\end{Highlighting}
\end{Shaded}

\begin{verbatim}
## Confusion Matrix and Statistics
## 
##                              DIABETES
## mean_glucose_blood_bg_over200  1  0
##                             1  2  2
##                             0 25 51
##                                           
##                Accuracy : 0.6625          
##                  95% CI : (0.5481, 0.7645)
##     No Information Rate : 0.6625          
##     P-Value [Acc > NIR] : 0.552           
##                                           
##                   Kappa : 0.0459          
##                                           
##  Mcnemar's Test P-Value : 2.297e-05       
##                                           
##             Sensitivity : 0.07407         
##             Specificity : 0.96226         
##          Pos Pred Value : 0.50000         
##          Neg Pred Value : 0.67105         
##              Prevalence : 0.33750         
##          Detection Rate : 0.02500         
##    Detection Prevalence : 0.05000         
##       Balanced Accuracy : 0.51817         
##                                           
##        'Positive' Class : 1               
## 
\end{verbatim}

\subparagraph{The mean value of blood glucose blood gas of at least 200
mg/dL has a good specificity of 96\%. However the sensitivity is
exceptionally poor at only
7.41\%.}\label{the-mean-value-of-blood-glucose-blood-gas-of-at-least-200-mgdl-has-a-good-specificity-of-96.-however-the-sensitivity-is-exceptionally-poor-at-only-7.41.}

\subsubsection{Querying and Assessing the comibination of metformin and
insulin}\label{querying-and-assessing-the-comibination-of-metformin-and-insulin}

\begin{Shaded}
\begin{Highlighting}[]
\CommentTok{\#Load the prescriptions table}
\NormalTok{prescriptions }\OtherTok{\textless{}{-}} \FunctionTok{tbl}\NormalTok{(con, }\StringTok{"mimic3\_demo.PRESCRIPTIONS"}\NormalTok{)}

\CommentTok{\#Identify the patients with metfomin and insulin prescriptions}
\NormalTok{metformin\_and\_insulin }\OtherTok{\textless{}{-}}\NormalTok{ prescriptions }\SpecialCharTok{\%\textgreater{}\%} 
  \FunctionTok{filter}\NormalTok{(}\FunctionTok{lower}\NormalTok{(DRUG) }\SpecialCharTok{\%like\%} \StringTok{"metformin"} \SpecialCharTok{|}
           \FunctionTok{lower}\NormalTok{(DRUG) }\SpecialCharTok{\%like\%} \StringTok{"insulin"}\NormalTok{) }\SpecialCharTok{\%\textgreater{}\%} 
  \FunctionTok{mutate}\NormalTok{(}\AttributeTok{metformin\_counter =} \FunctionTok{case\_when}\NormalTok{(}\FunctionTok{lower}\NormalTok{(DRUG) }\SpecialCharTok{\%like\%} \StringTok{"\%metformin\%"} \SpecialCharTok{\textasciitilde{}} \DecValTok{1}\NormalTok{,}
                                       \ConstantTok{TRUE} \SpecialCharTok{\textasciitilde{}} \DecValTok{0}\NormalTok{),}
         \AttributeTok{insulin\_counter =} \FunctionTok{case\_when}\NormalTok{(}\FunctionTok{lower}\NormalTok{(DRUG) }\SpecialCharTok{\%like\%} \StringTok{"\%insulin\%"} \SpecialCharTok{\textasciitilde{}} \DecValTok{1}\NormalTok{,}
                                     \ConstantTok{TRUE} \SpecialCharTok{\textasciitilde{}} \DecValTok{0}\NormalTok{)) }\SpecialCharTok{\%\textgreater{}\%} 
  \FunctionTok{group\_by}\NormalTok{(SUBJECT\_ID) }\SpecialCharTok{\%\textgreater{}\%} 
  \FunctionTok{summarise}\NormalTok{(}\AttributeTok{any\_metformin =} \FunctionTok{max}\NormalTok{(metformin\_counter, }\AttributeTok{na.rm =} \ConstantTok{TRUE}\NormalTok{),}
            \AttributeTok{any\_insulin =} \FunctionTok{max}\NormalTok{(insulin\_counter, }\AttributeTok{na.rm =} \ConstantTok{TRUE}\NormalTok{)) }\SpecialCharTok{\%\textgreater{}\%} 
  \FunctionTok{filter}\NormalTok{(any\_metformin }\SpecialCharTok{==} \DecValTok{1}\NormalTok{,}
\NormalTok{         any\_insulin }\SpecialCharTok{==} \DecValTok{1}\NormalTok{) }\SpecialCharTok{\%\textgreater{}\%} 
  \FunctionTok{mutate}\NormalTok{(}\AttributeTok{metformin\_and\_insulin =} \DecValTok{1}\NormalTok{)}
\CommentTok{\# Join with the training data}
\NormalTok{training }\SpecialCharTok{\%\textgreater{}\%} 
  \FunctionTok{left\_join}\NormalTok{(metformin\_and\_insulin) }\SpecialCharTok{\%\textgreater{}\%} 
  \FunctionTok{mutate}\NormalTok{(}\AttributeTok{metformin\_and\_insulin =} \FunctionTok{coalesce}\NormalTok{(metformin\_and\_insulin, }\DecValTok{0}\NormalTok{)) }\SpecialCharTok{\%\textgreater{}\%} 
  \FunctionTok{collect}\NormalTok{() }\SpecialCharTok{\%\textgreater{}\%} 
\CommentTok{\# Evaluate performance}
  \FunctionTok{getStats}\NormalTok{(metformin\_and\_insulin, DIABETES)}
\end{Highlighting}
\end{Shaded}

\begin{verbatim}
## Confusion Matrix and Statistics
## 
##                      DIABETES
## metformin_and_insulin  1  0
##                     1  2  0
##                     0 25 53
##                                           
##                Accuracy : 0.6875          
##                  95% CI : (0.5741, 0.7865)
##     No Information Rate : 0.6625          
##     P-Value [Acc > NIR] : 0.3658          
##                                           
##                   Kappa : 0.0958          
##                                           
##  Mcnemar's Test P-Value : 1.587e-06       
##                                           
##             Sensitivity : 0.07407         
##             Specificity : 1.00000         
##          Pos Pred Value : 1.00000         
##          Neg Pred Value : 0.67949         
##              Prevalence : 0.33750         
##          Detection Rate : 0.02500         
##    Detection Prevalence : 0.02500         
##       Balanced Accuracy : 0.53704         
##                                           
##        'Positive' Class : 1               
## 
\end{verbatim}

\subparagraph{The comibination of metformin and insulin has a perfect
specificity of 100\%. However the sensitivity is exceptionally poor at
only
7.41\%.}\label{the-comibination-of-metformin-and-insulin-has-a-perfect-specificity-of-100.-however-the-sensitivity-is-exceptionally-poor-at-only-7.41.}

\subsubsection{Querying and Assessing the comibination of ICD9 OR
Glucose\textgreater=200 OR
Insulin}\label{querying-and-assessing-the-comibination-of-icd9-or-glucose200-or-insulin}

\begin{Shaded}
\begin{Highlighting}[]
\CommentTok{\#Load tables}
\NormalTok{labevents }\OtherTok{\textless{}{-}} \FunctionTok{tbl}\NormalTok{(con, }\StringTok{"mimic3\_demo.LABEVENTS"}\NormalTok{)}
\NormalTok{d\_labitems }\OtherTok{\textless{}{-}} \FunctionTok{tbl}\NormalTok{(con, }\StringTok{"mimic3\_demo.D\_LABITEMS"}\NormalTok{)}
\CommentTok{\#Identify the patients with ICD}
\NormalTok{any\_t2d\_icd }\OtherTok{\textless{}{-}}\NormalTok{ diagnoses\_icd }\SpecialCharTok{\%\textgreater{}\%} 
  \FunctionTok{filter}\NormalTok{(ICD9\_CODE }\SpecialCharTok{\%in\%} \FunctionTok{c}\NormalTok{(}\StringTok{"25000"}\NormalTok{, }\StringTok{"25002"}\NormalTok{,}\StringTok{"25010"}\NormalTok{, }\StringTok{"25012"}\NormalTok{,}\StringTok{"25020"}\NormalTok{, }\StringTok{"25022"}\NormalTok{,}\StringTok{"25030"}\NormalTok{, }
                          \StringTok{"25032"}\NormalTok{,}\StringTok{"25040"}\NormalTok{, }\StringTok{"25042"}\NormalTok{,}\StringTok{"25050"}\NormalTok{, }\StringTok{"25052"}\NormalTok{,}\StringTok{"25060"}\NormalTok{, }\StringTok{"25062"}\NormalTok{,}
                          \StringTok{"25070"}\NormalTok{, }\StringTok{"25072"}\NormalTok{,}\StringTok{"25080"}\NormalTok{, }\StringTok{"25082"}\NormalTok{,}\StringTok{"25090"}\NormalTok{, }\StringTok{"25092"}\NormalTok{)) }\SpecialCharTok{\%\textgreater{}\%} 
  \FunctionTok{distinct}\NormalTok{(SUBJECT\_ID) }\SpecialCharTok{\%\textgreater{}\%} 
  \FunctionTok{mutate}\NormalTok{(}\AttributeTok{any\_t2d\_icd =} \DecValTok{1}\NormalTok{)}
\CommentTok{\#Identify the patients with Glucose\textgreater{}=200 }
\NormalTok{any\_glucose\_blood\_bg\_over200 }\OtherTok{\textless{}{-}}\NormalTok{ labevents }\SpecialCharTok{\%\textgreater{}\%} 
  \FunctionTok{inner\_join}\NormalTok{(d\_labitems, }\AttributeTok{by =} \FunctionTok{c}\NormalTok{(}\StringTok{"ITEMID"} \OtherTok{=} \StringTok{"ITEMID"}\NormalTok{), }\AttributeTok{suffix =} \FunctionTok{c}\NormalTok{(}\StringTok{"\_l"}\NormalTok{,}\StringTok{"\_d"}\NormalTok{)) }\SpecialCharTok{\%\textgreater{}\%} 
  \FunctionTok{filter}\NormalTok{(LABEL }\SpecialCharTok{==} \StringTok{"Glucose"}\NormalTok{,}
\NormalTok{         FLUID }\SpecialCharTok{==} \StringTok{"Blood"}\NormalTok{,}
\NormalTok{         CATEGORY }\SpecialCharTok{==} \StringTok{"Blood Gas"}\NormalTok{) }\SpecialCharTok{\%\textgreater{}\%} 
  \FunctionTok{group\_by}\NormalTok{(SUBJECT\_ID) }\SpecialCharTok{\%\textgreater{}\%} 
  \FunctionTok{mutate}\NormalTok{(}\AttributeTok{glucose\_blood\_bg\_over200\_marker =} \FunctionTok{case\_when}\NormalTok{(VALUENUM }\SpecialCharTok{\textgreater{}=} \DecValTok{200} \SpecialCharTok{\textasciitilde{}} \DecValTok{1}\NormalTok{,}
                                                     \ConstantTok{TRUE} \SpecialCharTok{\textasciitilde{}}\DecValTok{0}\NormalTok{)) }\SpecialCharTok{\%\textgreater{}\%} 
  \FunctionTok{summarise}\NormalTok{(}\AttributeTok{any\_glucose\_blood\_bg\_over200 =} \FunctionTok{max}\NormalTok{(glucose\_blood\_bg\_over200\_marker, }\AttributeTok{na.rm =} \ConstantTok{TRUE}\NormalTok{)) }\SpecialCharTok{\%\textgreater{}\%} 
  \FunctionTok{select}\NormalTok{(SUBJECT\_ID, any\_glucose\_blood\_bg\_over200)}
\CommentTok{\#Identify the patients with Insulin}
\NormalTok{any\_insulin }\OtherTok{\textless{}{-}}\NormalTok{ prescriptions }\SpecialCharTok{\%\textgreater{}\%} 
  \FunctionTok{filter}\NormalTok{(}\FunctionTok{lower}\NormalTok{(DRUG) }\SpecialCharTok{\%like\%} \StringTok{"insulin"}\NormalTok{) }\SpecialCharTok{\%\textgreater{}\%} 
  \FunctionTok{distinct}\NormalTok{(SUBJECT\_ID) }\SpecialCharTok{\%\textgreater{}\%} 
  \FunctionTok{mutate}\NormalTok{(}\AttributeTok{any\_insulin =} \DecValTok{1}\NormalTok{)}
\CommentTok{\#Join with the training data}
\NormalTok{training }\SpecialCharTok{\%\textgreater{}\%} 
  \FunctionTok{left\_join}\NormalTok{(any\_t2d\_icd) }\SpecialCharTok{\%\textgreater{}\%} 
  \FunctionTok{left\_join}\NormalTok{(any\_glucose\_blood\_bg\_over200) }\SpecialCharTok{\%\textgreater{}\%} 
  \FunctionTok{left\_join}\NormalTok{(any\_insulin) }\SpecialCharTok{\%\textgreater{}\%} 
  \FunctionTok{mutate}\NormalTok{(}\AttributeTok{any\_t2d\_icd =} \FunctionTok{coalesce}\NormalTok{(any\_t2d\_icd, }\DecValTok{0}\NormalTok{),}
         \AttributeTok{any\_glucose\_blood\_bg\_over200 =} \FunctionTok{coalesce}\NormalTok{(any\_glucose\_blood\_bg\_over200, }\DecValTok{0}\NormalTok{),}
         \AttributeTok{any\_insulin =} \FunctionTok{coalesce}\NormalTok{(any\_insulin, }\DecValTok{0}\NormalTok{)) }\SpecialCharTok{\%\textgreater{}\%} 
  \FunctionTok{mutate}\NormalTok{(}\AttributeTok{icd\_or\_glucose\_or\_insulin =} \FunctionTok{case\_when}\NormalTok{(any\_t2d\_icd }\SpecialCharTok{==} \DecValTok{1} \SpecialCharTok{|}
\NormalTok{                                                 any\_glucose\_blood\_bg\_over200 }\SpecialCharTok{==} \DecValTok{1} \SpecialCharTok{|}
\NormalTok{                                                 any\_insulin }\SpecialCharTok{==} \DecValTok{1} \SpecialCharTok{\textasciitilde{}} \DecValTok{1}\NormalTok{,}
                                               \ConstantTok{TRUE} \SpecialCharTok{\textasciitilde{}} \DecValTok{0}\NormalTok{)) }\SpecialCharTok{\%\textgreater{}\%} 
  \FunctionTok{collect}\NormalTok{() }\SpecialCharTok{\%\textgreater{}\%} 
\CommentTok{\# Evaluate performance}
  \FunctionTok{getStats}\NormalTok{(icd\_or\_glucose\_or\_insulin, DIABETES)}
\end{Highlighting}
\end{Shaded}

\begin{verbatim}
## Confusion Matrix and Statistics
## 
##                          DIABETES
## icd_or_glucose_or_insulin  1  0
##                         1 26 34
##                         0  1 19
##                                          
##                Accuracy : 0.5625         
##                  95% CI : (0.447, 0.6732)
##     No Information Rate : 0.6625         
##     P-Value [Acc > NIR] : 0.9761         
##                                          
##                   Kappa : 0.2473         
##                                          
##  Mcnemar's Test P-Value : 6.338e-08      
##                                          
##             Sensitivity : 0.9630         
##             Specificity : 0.3585         
##          Pos Pred Value : 0.4333         
##          Neg Pred Value : 0.9500         
##              Prevalence : 0.3375         
##          Detection Rate : 0.3250         
##    Detection Prevalence : 0.7500         
##       Balanced Accuracy : 0.6607         
##                                          
##        'Positive' Class : 1              
## 
\end{verbatim}

\subparagraph{The comibination of ICD9 OR Glucose\textgreater=200 OR
Insulin has a poor specificity of 35.9\%. However the sensitivity is
exceptionally high at only
96.3\%.}\label{the-comibination-of-icd9-or-glucose200-or-insulin-has-a-poor-specificity-of-35.9.-however-the-sensitivity-is-exceptionally-high-at-only-96.3.}

\subsubsection{Querying and Assessing the comibination of ICD9 AND
Glucose\textgreater=200 AND
Insulin}\label{querying-and-assessing-the-comibination-of-icd9-and-glucose200-and-insulin}

\begin{Shaded}
\begin{Highlighting}[]
\CommentTok{\#Join with the traing data}
\NormalTok{training }\SpecialCharTok{\%\textgreater{}\%} 
  \FunctionTok{left\_join}\NormalTok{(any\_t2d\_icd) }\SpecialCharTok{\%\textgreater{}\%} 
  \FunctionTok{left\_join}\NormalTok{(any\_glucose\_blood\_bg\_over200) }\SpecialCharTok{\%\textgreater{}\%} 
  \FunctionTok{left\_join}\NormalTok{(any\_insulin) }\SpecialCharTok{\%\textgreater{}\%} 
  \FunctionTok{mutate}\NormalTok{(}\AttributeTok{any\_t2d\_icd =} \FunctionTok{coalesce}\NormalTok{(any\_t2d\_icd, }\DecValTok{0}\NormalTok{),}
         \AttributeTok{any\_glucose\_blood\_bg\_over200 =} \FunctionTok{coalesce}\NormalTok{(any\_glucose\_blood\_bg\_over200, }\DecValTok{0}\NormalTok{),}
         \AttributeTok{any\_insulin =} \FunctionTok{coalesce}\NormalTok{(any\_insulin, }\DecValTok{0}\NormalTok{)) }\SpecialCharTok{\%\textgreater{}\%} 
  \FunctionTok{mutate}\NormalTok{(}\AttributeTok{icd\_and\_glucose\_and\_insulin =} \FunctionTok{case\_when}\NormalTok{(any\_t2d\_icd }\SpecialCharTok{==} \DecValTok{1} \SpecialCharTok{\&\&}
\NormalTok{                                                 any\_glucose\_blood\_bg\_over200 }\SpecialCharTok{==} \DecValTok{1} \SpecialCharTok{\&\&}
\NormalTok{                                                 any\_insulin }\SpecialCharTok{==} \DecValTok{1} \SpecialCharTok{\textasciitilde{}} \DecValTok{1}\NormalTok{,}
                                               \ConstantTok{TRUE} \SpecialCharTok{\textasciitilde{}} \DecValTok{0}\NormalTok{)) }\SpecialCharTok{\%\textgreater{}\%}
  \FunctionTok{collect}\NormalTok{() }\SpecialCharTok{\%\textgreater{}\%} 
\CommentTok{\# Evaluate performance}
  \FunctionTok{getStats}\NormalTok{(icd\_and\_glucose\_and\_insulin, DIABETES)}
\end{Highlighting}
\end{Shaded}

\begin{verbatim}
## Confusion Matrix and Statistics
## 
##                            DIABETES
## icd_and_glucose_and_insulin  1  0
##                           1  3  0
##                           0 24 53
##                                           
##                Accuracy : 0.7             
##                  95% CI : (0.5872, 0.7974)
##     No Information Rate : 0.6625          
##     P-Value [Acc > NIR] : 0.2802          
##                                           
##                   Kappa : 0.1421          
##                                           
##  Mcnemar's Test P-Value : 2.668e-06       
##                                           
##             Sensitivity : 0.1111          
##             Specificity : 1.0000          
##          Pos Pred Value : 1.0000          
##          Neg Pred Value : 0.6883          
##              Prevalence : 0.3375          
##          Detection Rate : 0.0375          
##    Detection Prevalence : 0.0375          
##       Balanced Accuracy : 0.5556          
##                                           
##        'Positive' Class : 1               
## 
\end{verbatim}

\subparagraph{The comibination of ICD9 AND Glucose\textgreater=200 AND
Insulin has a perfect specificity of 100\%. However the sensitivity is
exceptionally poor at only
11\%.}\label{the-comibination-of-icd9-and-glucose200-and-insulin-has-a-perfect-specificity-of-100.-however-the-sensitivity-is-exceptionally-poor-at-only-11.}

\subsubsection{Querying and Assessing the comibination of Insulin AND
(ICD OR Glucose
\textgreater=200)}\label{querying-and-assessing-the-comibination-of-insulin-and-icd-or-glucose-200}

\begin{Shaded}
\begin{Highlighting}[]
\CommentTok{\#Join with the traing data}
\NormalTok{training }\SpecialCharTok{\%\textgreater{}\%} 
  \FunctionTok{left\_join}\NormalTok{(any\_t2d\_icd) }\SpecialCharTok{\%\textgreater{}\%} 
  \FunctionTok{left\_join}\NormalTok{(any\_glucose\_blood\_bg\_over200) }\SpecialCharTok{\%\textgreater{}\%} 
  \FunctionTok{left\_join}\NormalTok{(any\_insulin) }\SpecialCharTok{\%\textgreater{}\%} 
  \FunctionTok{mutate}\NormalTok{(}\AttributeTok{any\_t2d\_icd =} \FunctionTok{coalesce}\NormalTok{(any\_t2d\_icd, }\DecValTok{0}\NormalTok{),}
         \AttributeTok{any\_glucose\_blood\_bg\_over200 =} \FunctionTok{coalesce}\NormalTok{(any\_glucose\_blood\_bg\_over200, }\DecValTok{0}\NormalTok{),}
         \AttributeTok{any\_insulin =} \FunctionTok{coalesce}\NormalTok{(any\_insulin, }\DecValTok{0}\NormalTok{)) }\SpecialCharTok{\%\textgreater{}\%} 
  \FunctionTok{mutate}\NormalTok{(}\AttributeTok{insulin\_and\_ICDorGlucose =} \FunctionTok{case\_when}\NormalTok{(any\_insulin }\SpecialCharTok{==} \DecValTok{1} \SpecialCharTok{\&\&}
\NormalTok{                                                (any\_glucose\_blood\_bg\_over200 }\SpecialCharTok{==}\DecValTok{1} \SpecialCharTok{|}\NormalTok{ any\_t2d\_icd }\SpecialCharTok{==} \DecValTok{1}\NormalTok{) }\SpecialCharTok{\textasciitilde{}} \DecValTok{1}\NormalTok{,}
                                              \ConstantTok{TRUE} \SpecialCharTok{\textasciitilde{}}\DecValTok{0}\NormalTok{)) }\SpecialCharTok{\%\textgreater{}\%}
  \FunctionTok{collect}\NormalTok{() }\SpecialCharTok{\%\textgreater{}\%} 
\CommentTok{\# Evaluate performance}
  \FunctionTok{getStats}\NormalTok{(insulin\_and\_ICDorGlucose, DIABETES)}
\end{Highlighting}
\end{Shaded}

\begin{verbatim}
## Confusion Matrix and Statistics
## 
##                         DIABETES
## insulin_and_ICDorGlucose  1  0
##                        1 18  7
##                        0  9 46
##                                           
##                Accuracy : 0.8             
##                  95% CI : (0.6956, 0.8811)
##     No Information Rate : 0.6625          
##     P-Value [Acc > NIR] : 0.005046        
##                                           
##                   Kappa : 0.5445          
##                                           
##  Mcnemar's Test P-Value : 0.802587        
##                                           
##             Sensitivity : 0.6667          
##             Specificity : 0.8679          
##          Pos Pred Value : 0.7200          
##          Neg Pred Value : 0.8364          
##              Prevalence : 0.3375          
##          Detection Rate : 0.2250          
##    Detection Prevalence : 0.3125          
##       Balanced Accuracy : 0.7673          
##                                           
##        'Positive' Class : 1               
## 
\end{verbatim}

\subparagraph{Using insulin alone had a specificity of 37.74\%, and a
sensitivity of 85.19\%.By requiring that patients with insuline must
also have a record of an ICD9 code or high glucose measurement we raised
the specificity to 86.79\%, and only dropped the sensitivity to
66.67\%.}\label{using-insulin-alone-had-a-specificity-of-37.74-and-a-sensitivity-of-85.19.by-requiring-that-patients-with-insuline-must-also-have-a-record-of-an-icd9-code-or-high-glucose-measurement-we-raised-the-specificity-to-86.79-and-only-dropped-the-sensitivity-to-66.67.}

\end{document}
